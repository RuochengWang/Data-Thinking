\section{数据与变量构建}

\subsection{数据来源与样本说明}

本文所使用的数据来源于整理后的中国招聘岗位数据集(\texttt{china\_job\_market\_2025.csv}),数据以招聘岗位为基本观测单元,涵盖岗位所属城市、行业类型、技能要求、学历要求以及薪资水平等信息。该数据能够较为全面地反映不同城市在劳动力需求结构与岗位特征方面的差异。

在数据处理过程中,本文以城市为核心分析对象,对岗位层面的信息进行聚合,构建城市层面的结构性指标。通过这一处理方式,可以有效减少个体岗位噪声对分析结果的干扰,同时更好地刻画城市整体的人才吸引力与产业结构特征。

在样本筛选方面,本文剔除了关键信息缺失的岗位记录,并对薪资变量进行了必要的数值清洗与标准化处理,以保证后续模型分析的稳定性与可解释性。

\subsection{城市层面变量构建思路}

考虑到直接在岗位层面引入大量分类变量容易导致模型复杂度过高以及多重共线性问题,本文在变量构建阶段即进行信息压缩与结构化处理,将岗位特征映射为具有明确经济含义的城市层面指标。具体而言,本文围绕以下三个核心维度构建城市特征变量:学历结构、高端产业结构以及技能结构多样性。

\subsection{学历结构变量}

学历水平是衡量城市高端人才集聚程度的重要指标。本文以岗位对博士学历(PhD)的需求作为高端学术与研究型人才的代表性特征,构建城市层面的博士需求占比变量。其定义如下:

$$
PhDRatio_c = \frac{\# (PhD)_c}{\# {card}(job)_c}
$$

该指标反映了城市对高层次学术或研发型人才的需求强度,在一定程度上也间接体现了城市的科研环境与高端创新活动水平。

\subsection{高端产业占比变量}

产业结构是影响城市薪资水平与人才吸引力的关键因素之一。相比逐一引入行业虚拟变量,本文采用“高端产业占比”这一综合性指标,以刻画城市在高附加值产业方面的集中程度。

基于数据集中已有的行业分类,本文将以下行业划分为高端或技术密集型产业:信息技术(IT)、工程(Engineering)、设计/IT(Design/IT)、金融(Finance)以及医疗健康(Healthcare)。上述行业通常具有较高的人力资本要求和薪资水平,能够较好地代表城市的高端产业结构。

据此,定义城市层面的高端产业占比为:

$$
High-tech\ Industry\ Ratio_c = \frac{\# (High-tech\ Industry)_c}{\# (job)_c}
$$

该变量反映了城市产业结构的技术密集程度,是后续分析城市薪资决定机制与城市类型差异的重要解释变量。

\subsection{技能结构与技能多样性指标}

除学历与产业结构外,岗位技能需求结构同样是衡量城市劳动力市场特征的重要维度。不同于仅关注某一特定技能是否出现,本文更关注城市整体技能需求的多样性水平。

数据集中每个岗位对应一组技能关键词,技能之间以分号或逗号分隔。本文首先对所有岗位的技能文本进行拆分与整理,构建完整的技能集合;随后,在城市层面对技能出现频率进行统计,计算各技能在城市岗位中的相对占比。

为量化技能结构的多样性,本文引入香农信息熵(Shannon Entropy)作为技能多样性指标。对于城市 \( c \),其技能香农指数定义为:

$$
Skill\ Entropy_c = - \sum_{i=1}^{N} p_{ci} \ln(p_{ci})
$$

其中,\( p_{ci} \) 表示技能 \( i \) 在城市 \( c \) 所有技能需求中所占的比例,\( N \) 为该城市出现的不同技能种类数量。

香农指数越大,说明城市的技能需求分布越均匀、技能结构越多样;反之,则表明城市岗位技能需求相对集中。该指标能够从整体层面反映城市对复合型人才和多技能劳动力的需求程度。

\subsection{薪资变量处理}

作为本文分析的核心结果变量,薪资水平在不同岗位与城市之间存在显著差异。由于薪资总体水平是右偏状态,本文对城市平均薪资取对数处理,构建城市层面的平均对数薪资指标:

$$
Avg\ Log\ Salary_c = \ln(Avg\ Salary_c)
$$

该处理方式在保留相对差异信息的同时,有助于缓解薪资分布右偏问题,适用于后续回归分析与机器学习建模。

\subsection{变量标准化处理}

在涉及多变量综合分析与聚类分析的模型中,不同变量的量纲与取值范围可能存在较大差异。为避免某一变量因尺度过大而主导模型结果,本文对所有城市层面特征变量进行了标准化处理。具体方法为对变量进行零均值、单位方差的标准化变换。

标准化后的变量主要用于模型二与模型三的分析,而在解释性分析中,本文仍保留原始变量的经济含义进行讨论。
