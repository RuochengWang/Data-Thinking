\section{模型设定与研究方法}

在前一章中,本文构建了刻画城市人才结构与产业特征的核心变量。本章在此基础上,进一步设计三类互补的分析模型,分别用于解释薪资决定机制、刻画高薪岗位特征以及对城市进行综合画像与分类。三类模型在研究目标、建模方法与解释侧重点上各有侧重,共同构成本文的整体分析框架。

\subsection{整体建模思路}

本文的研究目标并非单一的因果识别,而是通过多角度建模,系统刻画城市人才市场的结构特征及其与薪资水平之间的关系。为此,本文采用由“解释性建模”向“结构刻画与分类建模”逐步推进的策略,依次构建以下三类模型:

\begin{itemize}
    \item 模型一:基于机器学习回归的薪资决定机制分析,用于解释城市层面薪资差异的主要结构性来源;
    \item 模型二:高薪岗位的分类模型,用于识别哪些岗位特征更可能对应高薪工作;
    \item 模型三:城市人才吸引力与结构特征的聚类分析,用于对城市进行画像与分组。
\end{itemize}

这一建模路径由“解释薪资差异”逐步过渡到“刻画岗位结构”与“城市类型划分”,能够从不同层面回答本文提出的研究问题。

\subsection{模型一:薪资决定机制的解释性建模}

\subsubsection{模型目标}

模型一的研究目标是解释不同城市之间平均薪资水平差异的主要来源。与传统线性回归模型不同,本文并不试图估计严格的因果效应,而是关注在控制变量数量较少的情况下,哪些结构性因素在解释城市薪资差异时具有更高的重要性。

为避免多重共线性问题,同时提升模型的稳健性与拟合能力,本文采用基于树模型的机器学习回归方法,对城市平均对数薪资进行建模。(尝试过ols,共线性性非常严重,故更换了研究方法)

\subsubsection{模型设定}

设城市 \( c \) 的平均对数薪资为 \( y_c \),对应的解释变量向量为:
$$
\mathbf{X}_c = (PhDRatio_c,\ High-tech\ Industry\ Ratio_c,\ Skill\ Entropy_c)
$$

模型一可形式化表示为:
$$
y_c = f(\mathbf{X}_c) + \varepsilon_c
$$
其中,\( f(\cdot) \) 为非线性回归函数,通过机器学习方法进行估计,\( \varepsilon_c \) 表示随机误差项。

\subsubsection{变量重要性解释}

在模型估计完成后,本文通过特征重要性(feature importance)来衡量各解释变量在薪资决定中的相对贡献。变量重要性反映了在模型预测过程中,各变量对降低预测误差的贡献程度,从而为城市薪资差异提供结构性解释。

需要强调的是,该重要性指标用于描述解释能力而非因果强度,其结论应结合经济直觉进行解读。

\subsection{模型二:高薪岗位的分类建模}

\subsubsection{模型目标}

模型二从岗位层面出发,研究哪些岗位特征更可能对应高薪工作。与模型一关注城市整体薪资水平不同,模型二旨在刻画岗位之间的异质性,通过分类模型识别高薪岗位的典型特征组合。

该模型的核心研究问题为:在控制岗位经验要求、学历要求与城市层级等因素后,是否具备人工智能相关技能的岗位更有可能成为高薪岗位。

\subsubsection{高薪岗位定义}

本文以岗位薪资分布的分位数作为高薪岗位的判定标准。具体而言,当岗位的中位薪资高于样本薪资分布的第 75 百分位数时,将该岗位定义为高薪岗位,并赋值为 1;否则赋值为 0。记岗位 \( i \) 的高薪标签为:

$$
y_i =
\begin{cases}
1, & \text{若岗位 } i \text{的薪资大于75分位数} \\
0, & \text{否则}
\end{cases}
$$

该定义能够在保证样本规模的同时,突出薪资分布上部岗位的结构特征。

\subsubsection{解释变量设定}

模型二的解释变量主要包括岗位技能特征、岗位要求特征以及城市层级特征,具体包括:

\begin{itemize}
    \item \textbf{AI\_Skills}:核心解释变量,表示岗位是否要求人工智能相关技能。当岗位技能描述中同时包含 “Python”、“Machine Learning” 与 “Statistics” 时,该变量取值为 1,否则取值为 0。
    \item \textbf{Experience\_Level}:岗位经验要求,为分类变量,刻画岗位对工作年限与资历的要求。
    \item \textbf{Education\_Requirement}:岗位学历要求,为分类变量,反映人力资本门槛差异。
    \item \textbf{City\_Tier}:城市层级变量,用于刻画城市发展水平与劳动力市场环境差异。
\end{itemize}

其中,\textbf{City\_Tier} 并非数据集中直接给出的变量,而是基于城市的经济发展水平与市场成熟度进行人为划分。本文将一线及核心经济城市划分为 T1 城市,其余城市不做区分,因为数据集中城市只设计一二线城市。该划分方式在城市经济与劳动力市场研究中具有较强的现实解释意义,有助于控制城市层面不可观测的制度与市场环境差异。

所有分类变量在建模过程中均采用哑变量(dummy variables)形式进入模型。

\subsubsection{模型方法选择}

鉴于模型二属于二分类问题,本文采用逻辑回归(Logistic Regression)与基于树的集成学习方法(XGBoost)进行建模。

逻辑回归作为经典的广义线性模型,其参数具有明确的经济含义,能够直接刻画各解释变量对岗位成为高薪岗位概率的边际影响,因此适合作为基准模型,用于提供可解释性分析。

然而,逻辑回归对变量之间的线性可分性与函数形式假设较为严格。为进一步刻画变量之间潜在的非线性关系与交互效应,本文引入 XGBoost 分类模型。XGBoost 通过梯度提升树的方式,在预测性能与处理复杂结构方面具有显著优势,能够作为对逻辑回归结果的重要补充。

\subsubsection{模型评估指标}

在模型评估方面,本文综合采用准确率(Accuracy)、ROC-AUC 以及 F1-score 作为衡量标准。其中,F1-score 在类别分布不平衡的情况下尤为重要,有助于评估模型对高薪岗位的识别能力。


\subsection{模型三:城市人才吸引力与结构特征的聚类分析}

\subsubsection{模型目标}

模型三的目标是基于多维结构性指标,对城市进行综合画像与分类。与前两个模型关注预测或解释不同,模型三侧重于无监督学习,通过聚类方法识别在人才结构与产业特征方面具有相似性的城市类型。

该模型旨在回答以下问题:不同城市在人才结构与技能需求方面是否呈现出可区分的类型特征?

\subsubsection{特征向量构建}

在模型三中,每个城市 \( c \) 由以下标准化后的特征向量表示:
$$
\mathbf{w}_c = (Avg\ Log\ Salary_c,\ PhD\ Ratio_c,\ High-tech\ Industry\ Ratio_c,\ Skill\ Entropy_c)
$$

所有特征在进入聚类模型前均进行标准化处理,以避免尺度差异对聚类结果产生影响。

\subsubsection{聚类方法}

本文采用 K-means 聚类方法对城市进行分类。该方法通过最小化类内平方误差,将城市划分为 \( K \) 个互不重叠的类别。其优化目标为:
$$
\min \sum_{k=1}^{K} \sum_{c \in C_k} \lVert \mathbf{w}_c - \boldsymbol{\mu}_k \rVert^2
$$
其中,\( C_k \) 表示第 \( k \) 类城市集合,\( \boldsymbol{\mu}_k \) 为该类的中心向量。

为便于结果展示与解释,本文进一步采用主成分分析(PCA)将高维城市特征降至二维空间,并对聚类结果进行可视化分析。

\subsubsection{城市画像解释}

在完成聚类后,本文计算各城市类别在关键特征上的均值水平,形成城市画像。通过比较不同类别在薪资水平、学历结构、产业结构与技能多样性方面的差异,可以为理解城市人才吸引力与发展模式提供直观依据。
