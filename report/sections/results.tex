\section{实证结果与分析}

\subsection{模型一结果分析:城市薪资决定机制}

\begin{table}[H]
\centering
\caption{模型一:城市平均薪资决定机制的特征重要性}
\begin{tabular}{lcc}
\toprule
变量名称 & 特征重要性 & 相对贡献比例 \\
\midrule
高端产业占比(High-tech Industry Ratio) & 0.461 & 46.1\% \\
技能多样性指数(Skill Entropy) & 0.405 & 40.5\% \\
博士需求占比(PhD Ratio) & 0.134 & 13.4\% \\
\bottomrule
\end{tabular}
\end{table}


模型一采用基于树的机器学习回归方法,对城市层面的平均对数薪资进行解释性建模。模型的拟合结果显示,其决定系数 \(R^2\) 为 1.000,均方根误差(RMSE)为 0.000,表明模型在样本内能够完全解释城市平均薪资的变异。

从特征重要性结果来看,高端产业占比在薪资决定机制中发挥最为关键的作用,其重要性权重达到 46.1\%。该结果表明,城市中高端、技术密集型产业的集中程度是决定整体薪资水平的首要因素。这一发现与经济学中“产业结构升级推动薪资提升”的理论预期高度一致。

其次,技能多样性指数的重要性达到 40.5\%,显示城市对多技能、复合型人才的需求结构同样对薪资水平具有显著影响。技能结构越多样的城市,通常意味着岗位类型更加丰富,对高技能劳动力的竞争更为激烈,从而推高整体薪资水平。

相比之下,博士需求占比的相对重要性为 13.4\%,虽然仍具有正向影响,但其贡献程度明显低于产业结构与技能结构因素。这一结果说明,单纯依赖高学历人才集聚并不足以决定城市整体薪资水平,其作用需要通过产业环境与技能需求结构的匹配才能充分体现。

需要指出的是,模型一的拟合优度极高,反映出城市平均薪资在本研究框架下可被少数结构性变量高度解释。这一结果在一定程度上源于变量构建阶段对信息的高度聚合,使模型更偏向于结构解释而非预测外推。因此,模型一的主要价值在于揭示城市薪资差异的核心结构性因素,而非用于样本外预测。

\subsection{模型二结果分析:高薪岗位的分类特征}

模型二采用逻辑回归与 XGBoost 两种分类模型,对岗位是否属于高薪岗位进行预测。表 4.2 汇报了两种模型在测试集上的预测性能。从整体表现来看,两种模型在 Accuracy、ROC-AUC 与 F1-score 指标上均表现良好,且结果高度一致,表明模型设定具有较强的稳健性。

\begin{table}[H]
\centering
\caption{模型二:高薪岗位分类模型的预测性能比较}
\begin{tabular}{lccc}
\toprule
模型 & Accuracy & ROC-AUC & F1-score \\
\midrule
Logistic Regression & 0.829 & 0.861 & 0.707 \\
XGBoost             & 0.827 & 0.862 & 0.702 \\
\bottomrule
\end{tabular}
\end{table}

模型二采用逻辑回归与 XGBoost 两种分类模型,对岗位是否属于高薪岗位进行预测。表 4.2 汇报了两种模型在测试集上的预测性能。从整体表现来看,两种模型在 Accuracy、ROC-AUC 与 F1-score 指标上均表现良好,且结果高度一致,表明模型设定具有较强的稳健性。

具体而言,逻辑回归模型的 ROC-AUC 为 0.861,XGBoost 模型的 ROC-AUC 为 0.862,说明两种模型在区分高薪岗位与非高薪岗位方面均具备较强的判别能力。与此同时,F1-score 均在 0.70 左右,表明模型在高薪岗位这一相对少数类别上的识别效果较为理想。

为进一步分析各解释变量对模型预测结果的影响方向与相对重要性,本文基于 XGBoost 模型引入 SHAP(SHapley Additive exPlanations)方法进行可解释性分析。图 \ref{fig:shap_model2} 展示了各变量的 SHAP 值分布情况。

\begin{figure}[H]
\centering
\includegraphics[width=0.85\textwidth]{model_2_shap.png}
\caption{XGBoost 模型的 SHAP 特征重要性与影响方向}
\label{fig:shap_model2}
\end{figure}

从 SHAP 结果可以看出,岗位经验要求是影响高薪岗位判定的最关键因素。其中,Senior 级岗位对成为高薪岗位具有显著的正向贡献,而 Entry 级岗位则明显降低岗位进入高薪区间的概率。这一结果与劳动力市场中经验溢价的普遍认知高度一致。

需求指数(Demand Index)同样表现出较强的正向影响,说明市场需求旺盛的岗位更容易形成薪资溢价。这反映了劳动力供需关系在薪资形成机制中的重要作用。

人工智能技能变量(AI\_Skills)在模型中呈现出稳定的正向影响。具备 Python、机器学习与统计等复合技能要求的岗位,其 SHAP 值整体偏正,表明人工智能相关技能能够显著提升岗位成为高薪岗位的概率。这一发现从岗位层面验证了人工智能技能的薪资溢价效应。

在城市层级变量方面,一线城市(T1)岗位对高薪岗位判定具有正向贡献,而二线城市(T2)的影响相对有限,说明城市发展层级仍然在薪资形成中发挥重要的结构性作用。

相比之下,不同学历要求变量的 SHAP 值整体分布较为集中,其影响程度弱于经验、技能与需求因素。这一结果表明,在岗位层面的薪资分化中,学历更多体现为进入特定岗位的门槛条件,而非决定薪资高低的核心因素。

\subsection{模型三:城市高端就业结构的无监督聚类分析}

\subsubsection{聚类变量选取与数据处理}

模型三选取以下四个城市层面的核心变量作为聚类特征,分别从薪资水平、人才结构、产业结构与技能多样性四个维度刻画城市的高端就业特征:

(1)平均对数薪资(avg\_log\_salary),用于反映城市整体高端岗位的薪资水平;  

(2)博士学历占比(phd\_ratio),用于衡量城市对高层次学术与科研型人才的需求强度;  

(3)高端产业岗位占比(hi\_industry\_ratio),用于刻画城市产业结构中技术密集型行业的比重;  

(4)技能香农熵(skill\_entropy),用于衡量城市岗位技能需求结构的多样化程度。

为消除量纲差异对聚类结果的影响,本文对上述变量进行标准化处理,并在此基础上实施 K-Means 聚类分析。结合聚类稳定性与经济解释性,最终将样本城市划分为四类。

\subsubsection{不同城市类型的特征均值比较}

\begin{table}[H]
\centering
\caption{不同城市聚类的高端就业特征均值}
\label{tab:cluster_profile}
\begin{tabular}{ccccc}
\toprule
城市类别 & 平均对数薪资 & 博士占比 & 高端产业占比 & 技能熵 \\
\midrule
Cluster 0 & 1.377 & -0.296 & -1.959 & 1.542 \\
Cluster 1 & -0.029 & -0.636 & -0.413 & -0.517 \\
Cluster 2 & 0.398 & 1.022 & 0.869 & -0.199 \\
Cluster 3 & -1.584 & -0.182 & 0.253 & 1.538 \\
\bottomrule
\end{tabular}
\end{table}

表 \ref{tab:cluster_profile} 汇报了四类城市在核心特征变量上的均值情况(均为标准化值)。结果显示,不同城市类别在薪资水平、学历结构、产业结构与技能需求多样性方面均存在显著差异,表明城市高端就业结构具有明显的分层特征。

\subsubsection{不同城市类型的结构特征解释}

第一类城市(Cluster 0)在平均薪资水平与技能香农熵指标上显著高于其他类别,表明该类城市的高端岗位不仅薪资回报较高,同时对多样化技能组合具有较强需求。然而,该类城市在高端产业占比与博士学历需求方面相对较低,显示其高端就业优势更多来源于复合型技能与高附加值服务业。

第二类城市(Cluster 1)在薪资水平、博士学历占比、高端产业占比及技能熵等指标上均处于相对较低水平,表现出整体高端就业结构较为薄弱的特征。这类城市可能更多依赖传统产业或低附加值服务业,在高端岗位数量与质量方面均面临一定约束。

第三类城市(Cluster 2)在博士学历占比与高端产业岗位占比上显著高于其他类别,同时薪资水平亦处于中上区间,表明该类城市的高端就业结构高度依赖技术密集型产业,对科研能力与专业技术深度具有较强需求。

第四类城市(Cluster 3)在技能熵指标上表现出较高水平,但平均薪资水平显著偏低,高端产业占比与博士学历需求亦不突出。这一结果说明,技能需求的多样化本身并不必然转化为高薪回报,若缺乏高端产业支撑,技能结构的复杂性难以形成实质性的高端就业优势。

\subsubsection{城市类型的可视化分析}

为进一步直观展示不同城市类型在高端就业结构上的差异,本文基于聚类特征,采用主成分分析(PCA)方法对城市进行二维降维,并绘制城市类型分布图,如图 \ref{fig:pca_cluster} 所示。

\begin{figure}[H]
\centering
\includegraphics[width=0.85\textwidth]{城市画像_PCA聚类图.png}
\caption{基于技能多样性与产业结构的城市类型 PCA 可视化结果}
\label{fig:pca_cluster}
\end{figure}

从图 \ref{fig:pca_cluster} 可以看出,不同类别城市在主成分空间中呈现出较为清晰的分布结构,验证了聚类结果在整体结构上的合理性。

\subsection{三种模型结果的综合对照与一致性分析}

本文分别通过回归建模、分类建模与聚类分析三种不同建模范式,对高端就业结构的形成机制进行了多角度刻画。尽管三种模型在研究目标与方法上存在差异,但其核心结论在整体上具有较强的一致性。

首先,模型一基于回归框架,从连续变量角度刻画了城市层面高端就业特征对薪资水平的解释能力。结果显示,高端产业占比、技能多样性与高学历人才集聚对薪资水平具有显著影响,表明高端就业并非单一要素驱动,而是多维结构共同作用的结果。

其次,模型二通过分类建模,进一步验证了上述结论在岗位层面上的稳健性。无论是 Logistic 回归还是 XGBoost 模型,高端技能需求、工作经验与城市层级变量均在高薪岗位判别中发挥关键作用,说明产业结构与技能结构在高薪岗位形成中具有重要解释力。

最后,模型三通过无监督聚类方法,从结构分布角度揭示了城市之间在高端就业模式上的系统性差异。聚类结果表明,不同城市在高薪、高学历、高端产业与技能多样性之间呈现出多样化组合关系,与模型一、模型二所揭示的多维决定机制高度一致。

综上所述,三种模型从不同角度共同验证了高端就业结构的多维性与异质性特征,增强了本文研究结论的整体可信度,也为后续政策讨论与扩展研究奠定了坚实基础。

