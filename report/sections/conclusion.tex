\section{结论与政策启示}

\subsection{主要研究结论}

本文基于招聘岗位微观数据,从岗位层面与城市层面系统分析了我国高端就业结构的形成机制。通过构建回归模型、分类模型与聚类模型,本文从不同角度刻画了高端岗位薪资、高薪岗位判别以及城市高端就业结构的异质性特征,主要结论如下。

首先,从连续薪资决定机制的角度看,高端就业并非由单一因素驱动,而是由产业结构、人才结构与技能结构的多维因素共同决定。模型一的结果表明,高端产业岗位占比、技能需求多样性以及高学历人才集聚程度对城市薪资水平具有显著解释力,其中产业结构与技能结构的影响尤为突出。这一结论表明,高端就业优势的形成依赖于产业生态与技能结构的系统性协同,而非单纯依靠学历或人口规模。

其次,从岗位层面的高薪判别结果来看,高薪岗位在技能要求、经验门槛与城市区位等方面呈现出显著特征。模型二的分类结果显示,具备高端技能组合(如人工智能与数据分析相关技能)、中高等级工作经验以及位于高能级城市的岗位,更有可能被判定为高薪岗位。同时,机器学习模型在预测性能上与传统逻辑回归结果高度一致,增强了结论的稳健性。

最后,从城市整体结构的角度看,不同城市在高端就业结构上存在显著异质性。模型三的聚类分析表明,城市可以根据薪资水平、产业结构、人才结构与技能多样性被划分为若干具有不同发展模式的类型。一部分城市依托高端产业与科研能力形成技术密集型高端就业结构,另一部分城市则更多依赖技能多样化与服务业发展,而部分城市在上述维度上均相对薄弱,呈现出明显的结构性差距。

\subsection{政策启示}

基于本文的实证结果,可以从以下几个方面为城市高端就业发展提供政策启示。

第一,推动高端产业集聚是提升城市高端就业质量的关键路径。研究结果表明,高端产业占比在薪资决定与城市类型划分中均发挥核心作用。因此,地方政府应通过产业政策、科技创新支持与产业链完善,引导技术密集型产业在具备基础条件的城市形成集聚效应,从而提升高端岗位的数量与质量。

第二,重视技能结构而非单一学历指标。模型结果显示,技能多样性对高端就业具有重要影响,而学历本身并非唯一决定因素。这意味着,在人才政策设计中,应更加重视技能培养、跨领域能力与复合型人才的发展,通过职业培训、终身教育体系建设等方式,提升劳动者的技能适配能力。

第三,因城施策,避免“一刀切”的人才政策。模型三揭示了城市在高端就业结构上的显著异质性,不同城市适合的发展路径并不相同。对于技术基础雄厚的城市,应重点强化科研与高端制造能力;对于服务业与技能多样性较强的城市,则可通过提升产业附加值与岗位质量,实现高端就业的结构升级。

\subsection{研究不足与未来展望}

尽管本文在数据与方法上力求全面,但仍存在一定局限性。首先,受数据可得性限制,本文未能引入更为细致的企业层面信息,如企业规模、所有制性质等因素,未来研究可在此基础上进一步拓展分析维度。

其次,本文主要基于静态截面数据进行分析,尚未刻画高端就业结构的动态演化过程。未来研究可结合时间序列或面板数据,系统分析高端就业结构的演变趋势与路径依赖特征。

最后,本文对高端产业与技能的界定仍依赖于经验规则与文本提取方法,随着数据与技术条件的改善,后续研究可尝试引入更精细的自然语言处理与网络分析方法,以进一步提升测度精度。

总体而言,本文从多模型视角系统分析了高端就业结构的形成机制,为理解我国区域高端就业差异提供了有益参考,也为相关政策制定提供了实证依据。
