\section{引言}

近年来,随着区域经济一体化与城市竞争的不断加剧,不同城市在薪资水平、产业结构以及人才结构方面呈现出显著差异。城市之间在吸引高端人才、承载高附加值产业以及促进创新活动方面的能力差异,已成为影响区域经济长期发展的关键因素。因此,从数据角度系统刻画城市劳动力市场特征,并分析其背后的决定机制,具有重要的现实意义与政策价值。

传统的薪资决定机制研究多采用线性回归模型,在个体或岗位层面引入教育程度、行业类别、技能要求等大量解释变量。然而,在实际数据中,这类变量往往存在较强的相关性。例如,高端产业集聚程度较高的城市通常同时具有更高的学历水平和更丰富的技能结构,这容易导致模型中出现严重的多重共线性问题。尽管回归结果可能表现出较高的拟合优度,但参数估计的不稳定性和经济解释的可信度往往受到削弱。

针对上述问题,本文不再拘泥于个体层面的因果识别,而是转向城市层面的结构性分析,构建一套具有明确经济含义的城市特征指标体系,并结合机器学习与无监督学习方法,对中国城市劳动力市场进行多角度刻画。本文的核心目标在于回答以下三个问题:(1)城市层面的薪资水平主要由哪些关键结构性因素所决定?(2)那些岗位特征更可能对应高薪工作?(3)如何基于数据构建具有可解释性的城市画像,为比较分析与政策讨论提供依据?

为此,本文构建了三个相互关联但侧重点不同的模型。模型一聚焦于城市薪资决定机制,选取少量具有代表性的城市特征变量,并采用机器学习回归方法,在缓解多重共线性问题的同时,分析不同因素对薪资水平的重要性。模型二从分类角度出发,研究城市是否具备“高薪城市”特征,探讨城市结构变量在区分高薪与非高薪城市中的作用。模型三则进一步引入聚类分析方法,对城市进行综合画像刻画,识别不同类型城市在薪资水平、产业结构与技能多样性方面的系统性差异。

本文的研究特点主要体现在三个方面。首先,在变量构建层面,本文通过产业分类与技能文本处理,构建高端产业占比与技能多样性(香农指数\cite{shannon1948})等综合性指标,使城市特征更具概括性与经济解释力。其次,在方法层面,本文将机器学习方法引入薪资机制分析,在保持模型解释性的同时有效缓解多重共线性问题。最后,在研究视角上,本文以城市为分析单元,对中国劳动力市场进行结构化刻画,为城市比较研究提供了一种数据驱动的分析框架。

本文章节安排如下:第二章介绍数据来源与变量构建方法;第三章详细阐述模型设定与方法论;第四章给出实证结果与分析;第五章对研究结论进行总结,并讨论其政策含义与研究局限。
